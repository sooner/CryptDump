\section{implementation}

In this section, we describe briefly the problems we meet while doing experiments and the method we use the tackle those problems. The neweset version of cryptdb do not support STMT\citep{mysql-stmt} queries, and TPCC-MySQL uses STME to load data. We do not currently implement STMT for cryptdb, instead, we first insert plaintext data into cryptdb, dump data into a .sql file , and then insert data into cryptdb using this file to avoid using STMT. The tool mysqldump has default option of extended-insert\citep{mysqldump-extended-insert}, but if the size of insert query exceed a certain length, cryptdb can not support it. So we reconstruct the data to limit the size of extended insert. We only use warehousre two, and modify the data type of the tables that is not supported by cryptdb. Namely, we change date type to string type, float to integer type, signed integer to unsigned integer type. TPCC is the standard benchmark, so we still want to show the resutls based on it. The detailed configuration is aviliable at this site\citep{practical-cryptdb}.

For the backup choices, we should parse the metadata structure of cryptdb. For example, having the database name and table name, we should know such information as how many fields does the table have, how each field is replicated and encrypted, and the encrypted field name of each onion. All those information are stored in the database in the proxy, and can be parsed by reading the data and parsing the data. Those functions can be implemented in less than 800 lines of C++ Code, including the testing code. Our future work should make cryptdb stable and add new features to it, the backup code can also change due to that kind of change, but the underlining principal for backup is universal. To remove the layer rnd, we currently can only mannualy issue queries to cryptdb and then use the onion adjustment function to  accomplish that.

Current version of cryptdb also remove the layer ope-jion, we implemented it in our version of cryptdb and so the onion ope has three layers. The modified version and the related code are available at \citep{practical-cryptdb}.
